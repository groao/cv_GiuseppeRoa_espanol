%%%%%%%%%%%%%%%%%%%%%%%%%%%%%%%%%%%%%%%%%
% Friggeri Resume/CV
% XeLaTeX Template
% Version 1.0 (5/5/13)
%
% This template has been downloaded from:
% http://www.LaTeXTemplates.com
%
% Original author:
% Adrien Friggeri (adrien@friggeri.net)
% https://github.com/afriggeri/CV
%
% License:
% CC BY-NC-SA 3.0 (http://creativecommons.org/licenses/by-nc-sa/3.0/)
%
% Important notes:
% This template needs to be compiled with XeLaTeX and the bibliography, if used,
% needs to be compiled with biber rather than bibtex.
%%%%%%%%%%%%%%%%%%%%%%%%%%%%%%%%%%%%%%%%%
\documentclass[11pt]{friggeri-cv} % Add 'print' as an option into the square bracket to remove colors from this template for printing
%\addbibresource{bibliography.bib} % Specify the bibliography file to include publications
\usepackage[utf8]{inputenc}% \usepackage[latin1]{inputenc} 
\usepackage{fontawesome}
\usepackage{hyperref}
\definecolor{light-gray}{gray}{0.55}
\definecolor{skype}{HTML}{12A5F4}
\definecolor{html5}{HTML}{e34c26}
\definecolor{php}{HTML}{6c7eb7}
\definecolor{db}{HTML}{FF9900}
\definecolor{linkedin}{HTML}{1683BB}
\definecolor{white}{RGB}{255,255,255}
\definecolor{darkgray}{HTML}{333333}
\definecolor{gray}{HTML}{4D4D4D}
\definecolor{lightgray}{HTML}{999999}
\definecolor{green}{HTML}{088A08}
\definecolor{green2}{HTML}{088A08}
\definecolor{orange}{HTML}{FDA333}
\definecolor{yellow}{HTML}{FD7800}
\definecolor{purple}{HTML}{D3A4F9}
\definecolor{red}{HTML}{FB4485}
\definecolor{red1}{HTML}{FF0000}
\definecolor{blue}{HTML}{0000FF}
\definecolor{blue1}{HTML}{0080FF}
\definecolor{darkpurple}{HTML}{8904B1}
\definecolor{fondo}{HTML}{A4A4A4}
\definecolor{red2}{HTML}{B40404}
	
\begin{document}
\header{}{Giuseppe Roa Osorio}{LSSBB $\textregistered$ Lean Six Sigma Black Belt} {MSc Telecomunicaciones - Ingeniero Electrónico}
%----------------------------------------------------------------------------------------
%	SIDEBAR SECTION
%----------------------------------------------------------------------------------------
\begin{aside}
\section{Contacto}
{\color{black}{\faHome}} \color{headercolor}{Cll 146f $\#$73a-20 
Int 13 Apto 502
Bogota, Colombia}
{\color{green2}{\faWhatsapp}}  +57 320 964 1396
{\color{yellow}{\faEnvelope}}  \href{mailto:groao@unal.edu.co}{groao@unal.edu.co}
{\color{linkedin}{\faLinkedin}}  \href{https://www.linkedin.com/in/giuseppe-roa-osorio-lssbb-25341487}{giuseppe-roa-osorio}
~
\section{Idiomas}
Español Nativo
Francés Avanzado
(B2 DELF)
Inglés Intermedio
(B2 ITEP)
~
\section{Lenguajes {\color{php}\faCode} }
Phyton (+2 años)
R (+1 año)
Matlab (+1 año)
VB .Net (+2 años)
\LaTeX (+3 years)
%%----------------------------
~
\section{Bases de Datos {\color{gray}\faDatabase} }
Oracle (+2 años)
MySQL (+2 años)
%Cassandra (+1 año)
%%----------------------------
~
\section{Software Especializado}
{\color{php}\faSuperscript} Tableau, Minitab 
{\color{black}\faWindows} Microsoft Office
{\color{black}\faWrench} MapInfo, QGis
%{\color{black}\faSignal} NS-3, Atoll, NetAct 
%%----------------------------
\end{aside}
%%----------------------------------------------------------------------------------------
%%	PROFILE
%%----------------------------------------------------------------------------------------
\section{Perfil}

Ingeniero electrónico y Magíster en Telecomunicaciones de la Universidad Nacional de Colombia, con más de cinco años de experiencia en el sector de IT, reconocido como agente de cambio enfocado en la mejora continua y la toma de decisiones estratégicas basado en datos. En mi experiencia, he liderado equipos enfocados al análisis estadístico de indicadores comerciales y técnicos, la optimización de redes móviles, el desarrollo de soluciones software, la administración de bases de datos y sistemas de información geográfica.
%%----------------------------------------------------------------------------------------
%%	EDUCATION SECTION
%%----------------------------------------------------------------------------------------
\section{Formación}
\begin{entrylist}
%%------------------------------------------------
\entry
{En Curso}
{DS4A {\normalfont Colombia}}
{Correlation One, Bogotá}
{Entrenamiento intensivo de 10 semanas en Data Science.} 
%%------------------------------------------------
\entry
{2015--2019}
{Maestría {\normalfont en Telecomunicaciones}}
{Universidad Nacional de Colombia, Bogotá}
{Enfoque en redes modernas y análisis de Big Data.} 
%%------------------------------------------------
\entry
{2018}
{Diplomado {\normalfont Gestión, Gobierno y Arquitectura de Operación}}
{Netec, Bogotá}
{Énfasis en ITIL Foundation, COBIT, TOGAF, SCRUM}
%%------------------------------------------------
\entry
{2009--2014}
{Ingeniería {\normalfont Electrónica}}
{Universidad Nacional de Colombia, Bogotá}
{Énfasis  en sistemas digitales y gestión de proyectos IT.}
%%------------------------------------------------
\entry
{2012--2013}
{Intercambio académico {\normalfont en el programa SEI}}
{INPG PHELMA, Grenoble}
{Enfoque en Sistemas Electrónicos Integrados.}
%------------------------------------------------
%\entry
%{2003--2008}
%{Secundaria}
%{Fundación Ana Restrepo del Corral, Bogotá}%{}
%{Graduado con honores.}
%------------------------------------------------
\end{entrylist}
%%----------------------------------------------------------------------------------------
%%	WORK EXPERIENCE SECTION
%%----------------------------------------------------------------------------------------
\section{Experiencia}
%\begin{entrylist}
%------------------------------------------------
%\entry
%{Jul 2019- \newline
%Hoy}
%{Claro Colombia}
%{Bogotá, Colombia}
%{\emph{Líder de Analítica de Tecnología} \\
%Funciones y logros:\\
%$\bullet\quad$ Formación y gestión del equipo de analítica de tecnología orientado a la identificación reactiva y proactiva de oportunidades en la red convergente de Claro, la toma de decisiones estratégicas de inversión y el Data Story Telling de indicadores técnicos y de negocio claves.\\
%$\bullet\quad$ Análisis e interpretación de requerimientos de negocio de las diferentes áreas técnicas de la dirección corporativa de Tecnología.\\
%}
%%------------------------------------------------
%\end{entrylist} 
%\end{comment}
\begin{entrylist}
%------------------------------------------------
\entry
{Jul 2018- \newline
Hoy}
{Claro Colombia}
{Bogotá, Colombia}
{\emph{Ingeniero II Aseguramiento de Calidad} \\
%Funciones y logros:\\
$\bullet\quad$ Formación y gestión del equipo de analítica de tecnología orientado a la identificación reactiva y proactiva de oportunidades en la red convergente de Claro, la toma de decisiones estratégicas de inversión y el Data Story Telling de indicadores técnicos y de negocio claves.\\
$\bullet\quad$ Construcción modelo de evaluación de experiencia zonal en usuarios de redes móviles a partir de mediciones de proveedores como Facebook y Nokia orientado a identificar zonas potenciales para mejoras de servicio.\\
$\bullet\quad$ Construcción de modelo clasificador de dispositivos conectados en los hogares de usuarios de internet fijo encaminado a la identificación de oportunidades de explotación comercial.\\
%$\bullet\quad$ Desarrollo de procesos y planes de mejora, a partir de análisis estadísticos de tendencias de consumo e indicadores técnicos, orientados a beneficiar la experiencia de los usuarios de telefonía móvil e Internet fijo.\\
$\bullet\quad$ Análisis de impacto de diversas estrategias comerciales y de mejora de red empleando el toolkit de Lean Six Sigma, resumiendo los principales hallazgos y oportunidades de mejor continua en reportes ejecutivos en Tableau.\\
}
%%------------------------------------------------
\end{entrylist} 
\begin{entrylist}
%------------------------------------------------
\entry
{Dic 2015- \newline
Jul 2018}
{Claro Colombia}
{Bogotá, Colombia}
{\emph{Ingeniero II Sistemas de Gestión de Red} \\
%Funciones y logros:\\
$\bullet\quad$ Diseño de scripts en sistemas de gestión como NetAct/U2000, solucionando automáticamente el 5\% de las fallas más comunes.\\
$\bullet\quad$ Líder Técnico para la integración de la gestión de fallas mediante los protocolos CORBA/SNMP entre múltiples NMS y un sistema experto, disminuyendo de 3-4 semanas el tiempo promedio por cada configuración.
}
%%------------------------------------------------
\end{entrylist} 
\begin{entrylist}
%------------------------------------------------
\entry
{Jul 2014- \newline
Dic 2015}
{Ericsson Colombia S.A}
{Bogotá, Colombia}
{\emph{Técnico de Radio Frecuencia para el proyecto de optimización de Telefónica} \\
%Funciones y logros:\\
$\bullet\quad$ Desarrollo de herramientas software en VB.Net para optimizar redes celulares, agilizando la identificación de parámetros incorrectos en EB.\\
$\bullet\quad$ Simulación de redes móviles 2G/3G/4G en el software Atoll, obteniendo mapas detallados de cobertura usados para proveer esta información al ente regulador, los suscriptores, y el departamento de planeación.\\
$\bullet\quad$ Diseño de base de datos para monitorear el despliegue de antenas, ahorrando 40\% del tiempo para generar reportes e identificar tareas críticas.
}
%%------------------------------------------------
\end{entrylist} 
\begin{entrylist}
%%------------------------------------------------
\entry
{Ene 2014- \newline 
Jul 2014}
{Telefónica (Colombia Telecomunicaciones S.A)}
{Bogotá, Colombia}
{\emph{Pasantía durante 6 meses en la Jefatura de Ingeniería Radio Frecuencia} \\
%Funciones y logros:\\
$\bullet\quad$ Diseño de reportes, en Excel/QlikView, con indicadores de desempeño de la red celular, facilitando el análisis de macro-comportamientos.\\
$\bullet\quad$ Desarrollo de macros en Excel para automatizar la configuración de parámetros de radio acceso para la implementación de nuevas estaciones, ahorrando tiempo y verificando los parámetros para tecnologías 2G/3G.
}
%%------------------------------------------------
\end{entrylist} 
%----------------------------------------------------------------------------------------
%	CERTIFICACIONES SECTION
%----------------------------------------------------------------------------------------
\section{Certificaciones - Cursos}
%2019 \hspace{0.8cm} {\textbf{Tableau} \hspace{1.3cm} Desktop Certified Associate}\\
2019 \hspace{0.6cm} {\textbf{Coursera} \hspace{1.3cm} Machine Learning in Python}\\
2019 \hspace{0.6cm} {\textbf{Coursera} \hspace{1.3cm} Deep Learning for Business}\\
2019 \hspace{0.6cm} {\textbf{Coursera}} \hspace{1.3cm} Financial Markets \\
2019 \hspace{0.6cm} {\textbf{Coursera}} \hspace{1.3cm} Introducción a DataScience: Programación Estadística R\\
2018 \hspace{0.6cm} {\textbf{IASSC}} \hspace{1.7cm} LSSBB Lean Six Sigma Black Belt \\
2018 \hspace{0.6cm} {\textbf{SCRUMstudy}} \hspace{0.3cm}  SFC Scrum Fundamentals\\
2014 \hspace{0.6cm} {\textbf{Forsk}} \hspace{1.95cm}  Atoll 3.2\\
2010 \hspace{0.6cm} {\textbf{ITTalent}} \hspace{1.3cm}  Excel Avanzado
%%---------------------------------------------------------------------------
%----------------------------------------------------------------------------------------
%	SIDEBAR SECTION 2nd PAGE
%----------------------------------------------------------------------------------------
\begin{aside2}
%\section{UML-BPMN} 
%{\color{gray}\faSitemap} Visual Paradigm 
%{\color{gray}\faSlack} LucidChart
%%----------------------------
~
\section{Líneas de Interés}
Telecomunicaciones
Internet of Things 
Algorithmic Trading
Data Mining
~
\section{Distinciones}
{\color{black}\faTrophy} Beca Maestría
Universidad Nacional
de Colombia (\href{https://www.ingenieria.bogota.unal.edu.co/facultad/consejo-de-facultad/actas-casos-estudiantiles/category/265-2015.html?download=1777:acta-consejo-facultad-013-2015-08-13}{Resolución 13/2015})
~
{\color{black}\faTrophy} Mejores promedios
Universidad Nacional
de Colombia (Ingeniería - 2011)
\end{aside2}
%%----------------------------------------------------------------------------------------
%%	PUBLICATIONS SECTION
%%----------------------------------------------------------------------------------------
%\pagebreak
\section{Publicaciones}
{$\bullet\quad$ Roa-Osorio, G. et al., “An address allocation protocol in ad hoc networks based on a pollen dispersion algorithm” in {\itshape{2nd IEEE Colombian conference on Computational Intelligence.}} Barranquilla, Colombia, June 2019. DOI 10.1109/ColCACI.2019.8781966.\\\\}
%{$\bullet\quad$ G. Roa and J. Ortiz, “Seguridad y regulación en servicios tipo máquina-máquina (ecall, ehealth)” in {\itshape{CICOM 2015 5o Congreso internacional de computación México-Colombia}}. Cartagena de Indias (Colombia), Sept 2015, pp. 357–363. ISSN 2462-9588.\\} 
%------------------------------------------------
%{$\bullet\quad$ Giuseppe Roa, Le-Pelleter T., Bonvilain A., Fesquet L. and  Chagoya A. 2014. Designing ultra-low power systems with non-uniform sampling and event-driven logic. In {\itshape{Proceedings of the 27th Symposium on Integrated Circuits and Systems Design}} (SBCCI '14). ACM, New York, NY, USA, Article 5, 6 pages.  DOI=10.1145/ 2660540.2660973. ISBN: 978-1-4503-3156-2.} 
{$\bullet\quad$ Giuseppe Roa, et al. 2014. Designing ultra-low power systems with non-uniform sampling and event-driven logic. In {\itshape{Proceedings of the 27th Symposium on Integrated Circuits and Systems Design}} ACM, New York, NY, USA, Article 5, 6 pages. ISBN: 978-1-4503-3156-2.\\}
%----------------------------------------------------------------------------------------
%\section{Referencias Laborales}
%\begin{itemize}
%\item{
%\textbf{Jhudy Consuelo Bolivar Cañon} \quad {Claro Colombia}\\
%{Gerente Aseguramiento de Calidad.\\
%{\color{green2}{\faWhatsapp}}  +57 310 265 8975 \qquad \qquad \qquad \quad {\color{yellow}{\faEnvelope}}  \href{mailto:jhudy.bolivar@claro.com.co}{jhudy.bolivar@claro.com.co}}}
%-----------------------------------------------
%\item{
%\textbf{Juan Carlos López Candamil} \quad {Claro %Colombia}\\
%{Gerente Sistemas de Gestión de Red.\\
%{\color{green2}{\faWhatsapp}}  +57 320 964 0055 \qquad \qquad \qquad \quad {\color{yellow}{\faEnvelope}}  \href{mailto:Juan.Lopezc@claro.com.co}{juan.lopezc@claro.com.co}}}
%-----------------------------------------------
%\item{
%\textbf{Javier Guillermo Pinto} \qquad \qquad {Telefónica Colombia}\\
%{Gerente Regional de Ingeniería RF.\\
%{\color{green2}{\faWhatsapp}}  +57 316 210 9982 \qquad \qquad \qquad \quad {\color{yellow}{\faEnvelope}}  \href{mailto:javier.pinto@telefonica.com}{javier.pinto@telefonica.com}}}
%-----------------------------------------------
%\item{
%\textbf{Laurent Fesquet} \qquad \qquad \qquad \quad {TIMA Laboratory}\\
%{CIS group Coodirector.\\
%{\color{green2}{\faWhatsapp}}  +33 476 574 812 \qquad\qquad\qquad\quad \enspace {\color{yellow}{\faEnvelope}}  \href{mailto:Laurent.Fesquet@imag.fr}{Laurent.Fesquet@imag.fr}}}
%\\
%\end{itemize}
%----------------------------------------------------------------------------------------
%	PERSONAL REFERENCES
%----------------------------------------------------------------------------------------
%\section{Referencias Personales}
%\begin{itemize}
%\item{
%\textbf{Germán Alfonso Tellez} \qquad \qquad {Claro Colombia}\\
%{Ingeniero II Sistemas de Gestión de Red.\\
%{\color{green2}{\faWhatsapp}}  +57 311 221 4641 \qquad \qquad \qquad \quad {\color{yellow}{\faEnvelope}}  \href{mailto:german.tellez@claro.com.co}{german.tellez@claro.com.co}}}
%-----------------------------------------------
%\item{
%\textbf{Mauricio Salgado Manrique} \qquad {Telefónica Colombia}\\
%{Ingeniero Senior RF.\\
%{\color{green2}{\faWhatsapp}}  +57 316 210 0239 \qquad \qquad \qquad \quad {\color{yellow}{\faEnvelope}}  \href{mailto:mauricio.salgado@telefonica.com}{mauricio.salgado@telefonica.com}}}
%-----------------------------------------------
%\item{
%\textbf{Sebastian Corzo} \qquad \qquad {Siemens Colombia}\\
%{Ingeniero Electrónico\\
%{\color{green2}{\faWhatsapp}}  +57 300 833 5930 \qquad \qquad \qquad \quad %{\color{yellow}{\faEnvelope}}  \href{mailto:scorzog@unal.edu.co}{scorzog@unal.edu.co}}}
%\end{itemize}
%----------------------------------------------------------------------------------------
\end{document}